\documentclass[11pt, a4paper]{article} %\documentclass[11pt, a4paper, twoside, openright]{article} %draft

\usepackage{graphicx,color}
\usepackage{amssymb, amsmath, array}
\usepackage{hyperref}
\usepackage{newpxtext, newpxmath}
\usepackage{minted}
\usepackage{xcolor}
\usepackage{csquotes}
\usemintedstyle{friendly}
\usepackage[parfill]{parskip}

\begin{document}

\onecolumn

% Example of title page for the projects carried out within DEDIS
% Copied from lasec 

% Simply include it in your mastex tex file: 
%        % Example of title page for the projects carried out within DEDIS
% Copied from lasec 

% Simply include it in your mastex tex file: 
%        % Example of title page for the projects carried out within DEDIS
% Copied from lasec 

% Simply include it in your mastex tex file: 
%        \input{cover}


% Updated October 2016


\newcommand{\logoepfl}[0]{
  \begin{center}
    \includegraphics[width=4cm]{EPFLlogo.pdf}
  \end{center}
  \hrule
}
\newcommand{\project}[1]{
  \begin{center}
    \large{#1}
  \end{center}
  \vspace{1cm}
}
\newcommand{\department}[1]{
  \begin{center}
    \large{#1}
  \end{center}
}
\newcommand{\lab}[1]{
  \begin{center}
    \large{#1}
  \end{center}
}
\newcommand{\supervisor}[3]{
  \begin{center}
    \begin{normalsize}{
        \bf #1}\\#2\\#3
    \end{normalsize}
  \end{center}
}
\renewcommand{\author}[1]{
  \begin{center}
    \Large{#1}
  \end{center}
  \vspace{0.5cm}
}
\renewcommand{\title}[1]{
  \vspace{3cm}
  \begin{center}
    \huge{#1}
  \end{center}
  \vspace{1.7cm}
}
\renewcommand{\date}[2]{
  \begin{center}
    \normalsize{#1 #2}
  \end{center}
  \vspace{0.5cm}
}

\thispagestyle{empty}


% begin title page
\logoepfl

\title{Constant-Time Big Numbers (for Go)}

\author{Lúcás Críostóir Meier}
\department{School of Computer and Communication Sciences}
\lab{Decentralized and Distributed Systems lab (DEDIS)}
\project{BSc Semester Project}

\date{June}{2021}

\begin{center}
  \supervisor{Responsible and Supervisor}{Prof. Bryan Ford}{EPFL / DEDIS}
\end{center}

% end title page




% Updated October 2016


\newcommand{\logoepfl}[0]{
  \begin{center}
    \includegraphics[width=4cm]{EPFLlogo.pdf}
  \end{center}
  \hrule
}
\newcommand{\project}[1]{
  \begin{center}
    \large{#1}
  \end{center}
  \vspace{1cm}
}
\newcommand{\department}[1]{
  \begin{center}
    \large{#1}
  \end{center}
}
\newcommand{\lab}[1]{
  \begin{center}
    \large{#1}
  \end{center}
}
\newcommand{\supervisor}[3]{
  \begin{center}
    \begin{normalsize}{
        \bf #1}\\#2\\#3
    \end{normalsize}
  \end{center}
}
\renewcommand{\author}[1]{
  \begin{center}
    \Large{#1}
  \end{center}
  \vspace{0.5cm}
}
\renewcommand{\title}[1]{
  \vspace{3cm}
  \begin{center}
    \huge{#1}
  \end{center}
  \vspace{1.7cm}
}
\renewcommand{\date}[2]{
  \begin{center}
    \normalsize{#1 #2}
  \end{center}
  \vspace{0.5cm}
}

\thispagestyle{empty}


% begin title page
\logoepfl

\title{Constant-Time Big Numbers (for Go)}

\author{Lúcás Críostóir Meier}
\department{School of Computer and Communication Sciences}
\lab{Decentralized and Distributed Systems lab (DEDIS)}
\project{BSc Semester Project}

\date{June}{2021}

\begin{center}
  \supervisor{Responsible and Supervisor}{Prof. Bryan Ford}{EPFL / DEDIS}
\end{center}

% end title page




% Updated October 2016


\newcommand{\logoepfl}[0]{
  \begin{center}
    \includegraphics[width=4cm]{EPFLlogo.pdf}
  \end{center}
  \hrule
}
\newcommand{\project}[1]{
  \begin{center}
    \large{#1}
  \end{center}
  \vspace{1cm}
}
\newcommand{\department}[1]{
  \begin{center}
    \large{#1}
  \end{center}
}
\newcommand{\lab}[1]{
  \begin{center}
    \large{#1}
  \end{center}
}
\newcommand{\supervisor}[3]{
  \begin{center}
    \begin{normalsize}{
        \bf #1}\\#2\\#3
    \end{normalsize}
  \end{center}
}
\renewcommand{\author}[1]{
  \begin{center}
    \Large{#1}
  \end{center}
  \vspace{0.5cm}
}
\renewcommand{\title}[1]{
  \vspace{3cm}
  \begin{center}
    \huge{#1}
  \end{center}
  \vspace{1.7cm}
}
\renewcommand{\date}[2]{
  \begin{center}
    \normalsize{#1 #2}
  \end{center}
  \vspace{0.5cm}
}

\thispagestyle{empty}


% begin title page
\logoepfl

\title{Constant-Time Big Numbers (for Go)}

\author{Lúcás Críostóir Meier}
\department{School of Computer and Communication Sciences}
\lab{Decentralized and Distributed Systems lab (DEDIS)}
\project{BSc Semester Project}

\date{June}{2021}

\begin{center}
  \supervisor{Responsible and Supervisor}{Prof. Bryan Ford}{EPFL / DEDIS}
\end{center}

% end title page




\thispagestyle{empty}

\newpage

\tableofcontents{\protect\thispagestyle{empty}

\clearpage

\section{Introduction}

With the rise of the internet and widespread
digital services, the importance of
secure communication has never been greater.
Thankfully, after 50 years of Public Key cryptography
\cite{hellman_overview_1978},
we have good theoretical tools to provide this security.

Most of these systems rely on modular arithmetic with large numbers.
For example, RSA \cite{rivest_method_1978},
or Elliptic Curve cryptography \cite{miller_use_1986}.
Working with these numbers is not natively supported by hardware.
Instead, we use a \enquote{Big Number}
software library to provide this functionality.

Unfortunately, although Public Key cryptosystems have been
thoroughly
scrutinized in theory,
implementations often suffer from vulnerabilities in practice.

One important class of vulnerability are
timing side-channels
\cite{kocher_cryptanalysis_1995}.
This is when an implementation leaks information
about secret values through its execution time.
Big Number libraries designed without cryptography in mind
suffer from these vulnerabilities.

In particular, Go \cite{the_go_authors_go_nodate} provides a general purpose
Big Number type, \texttt{big.Int},
which does not provide constant-time operations.
Unfortunately, this library
gets used for cryptography
\cite{ford_proposal_2017}, including inside of Go's own standard library,
in the \texttt{go/crypto} package.

We've addressed this issue by creating a library
\cite{meier_cronokirbysafenum_2021}
providing Big Numbers with constant-time operations.
Our library provides the necessary operations for Public Key cryptography,
while avoiding timing side-channels.
To demonstrate its utility, we've modified Go's \texttt{go/crypto}
package, replacing the use of \texttt{big.Int} in the DSA and RSA
systems, achieving a slowdown of only 2x for the latter.

\section{Background}

In this section, we explain how Big Numbers are used in Public Key
cryptography, what timing side-channels are, how we model their threat,
as well as what kind of vulnerabilities are present in Go's
\texttt{big.Int} type.

\subsection{Big Numbers in Cryptography}

Most Public Key cryptosystems rely on
modular arithmetic.

In RSA \cite{rivest_method_1978}, a public key $(e, N)$ consists
of a modulus
$N \in \mathbb{N}$, usually 2048 bits long, and an exponent $e
\in [0, \varphi(N) - 1]$.
To encrypt a message $m \in [0, N - 1]$, we calculate
$$
c := m^e \mod N
$$

Since $N$ is much larger than a register, 
we need a Big Number library to implement this system.
Modular exponentiation is also not implemented in hardware,
requiring extra support in software.

DSA
\cite{technology_digital_1994} also relies on modular arithmetic,
this time using a large prime $p$ of around 2048 bits, and working
in the multiplicative group $(\mathbb{Z}/p\mathbb{Z})^*$.

Elliptic Curve Cryptography
\cite{miller_use_1986} relies on complex formulas for adding points
on an elliptic curve, built over a finite field $K$. This field
is usually either a prime field $\mathbb{Z}/p\mathbb{Z}$, in which case arithmetic
modulo $p$ is used, or a
binary extension field $\text{GF}(2^n)$, in which case
binary and polynomial arithmetic are used.
For prime fields, Big Number functionality is necessary, because
the size of the prime is greater than 200 bits.

\subsubsection{Implementing Big Numbers}

When a modulus is known in advance, a special purpose
library implementing arithmetic with this fixed modulus
can be used. This is the case for Elliptic Curve cryptography,
where the prime field is a fixed parameter of the system.
Using a fixed type makes it easier to provide constant-time operation.

A disadvantage is that different systems require different moduli,
so it takes more work to implement and support each system.
One way to address this is to automatically generate implementations
of modular arithmetic, as done by FiatCrypto
\cite{hvass_high-assurance_nodate}.

In some systems, you need support for dynamic moduli.
Take RSA, for example.
In this case, you need a library that provides
dynamically sized numbers.

\subsubsection{Big Numbers in \texttt{go/crypto}}

Go provides implementations of the Public Key cryptosystems
we've mentioned so far in its \texttt{go/crypto} package.

Unfortunately \cite{ford_proposal_2017}, the general purpose
\texttt{big.Int} is used, in part, to implement these systems,
despite its potential vulnerability to timing attacks.

For DSA \cite{technology_digital_1994}, Go uses \texttt{big.Int}
for all operations, including key generation, signing, and verification.

For RSA \cite{rivest_method_1978}, Go embeds \texttt{big.Int}
as part of the API for
this package. Key generation, encryption, decryption, signing,
and verification all use \texttt{big.Int}.

For ECC \cite{miller_use_1986}, Go defines a general interface
for Elliptic Curves, requiring operations like point addition,
scalar multiplication, etc. All of these are defined in terms
of \texttt{big.Int}. Only \texttt{P384}
uses \texttt{big.Int} for field arithmetic directly. The other
curves instead convert \texttt{big.Int} to an internal
type for their field elements,
and perform their operations using that type instead. 
These internal types vary a lot between these curves.
\texttt{P224} uses hand-written operations in pure Go
for its field,
\texttt{P256} uses an optimized implementation in assembly,
and \texttt{P521} uses field arithmetic generated
by FiatCrypto
\cite{hvass_high-assurance_nodate}.

\subsection{Timing Attacks}

A side-channel
\cite{kelsey_side_1998}
leaks secret information indirectly,
through the observable effects of a program's execution.
Timing side-channels use the execution time of a program
to infer information about secret data it handles. 
A timing attack is the use of a timing side-channel to break the security
of some program or cryptographic algorithm.

When a program takes a different number of steps based on the value of some
secret, this is an obvious timing side-channel.
For example, a naive algorithm for comparing inputs with a secret password
might stop as soon as a mismatch is found. This algorithm has
an inherent timing side-channel. This can be exploited,
allowing the secret password to be guessed byte-by-byte.

Not all timing side-channels are this simple. Algorithms
that always take the same number of steps
can still having timing side-channels because of the underlying
hardware. For example, a processor may
execute an operation faster for some inputs, or the presence of a cache
could be used to infer what addresses are being accessed. These
microarchitectural timing side-channels are also of concern.
See \cite{ge_survey_2018} for a survey of these vulnerabilities.

\subsubsection{Actual Attacks}

Although the presence of a side-channel
does not directly lead to attacks,
Paul Kocher
demonstrated the potential for timing side-channels to break
crypto-systems as early as 1995
\cite{kocher_cryptanalysis_1995, kocher_timing_1996}.
These attacks relied on algorithms that perform a varying number of
operations based on secret data. 

One objection to timing attacks is that
while a timing side-channel is catastrophic in theory, in practice
this channel is too noisy to exploit. Unfortunately,
it's possible to exploit these attacks, even across
a network \cite{brumley_remote_2005, brumley_remote_2011}.
Noise only makes the channel more difficult to exploit,
requiring more samples to detect the underlying signal.

The use of caches as a potential side-channel was
identified early on as well \cite{page_theoretical_2002}.
Accessing data takes longer when that data is outside
of the cache.
If data accesses depend on a secret value, the observed execution
time will also depend on this value. Additionally,
an attacker located on the same machine
can place data into the cache and probe it themselves,
learning more precise information about the
program's access patterns. This kind of colocation is increasingly common,
as more applications are run on cloud servers.

A wide variety of attacks involving caches have been mounted
against various cryptosystems
\cite{
  bernstein_cache-timing_2005,
  yarom_cachebleed_2017,
  cabrera_aldaya_cache-timing_2019}
: accessing data based on secret values should
be avoided in cryptographic code.

\subsubsection{Our Threat-Model}
\label{threat_model}

These side-channels can be distilled into a simple,
albeit pessimistic, set of rules:

\begin{enumerate}
  \item Any loop leaks the number of iterations taken.
  \item Any memory access leaks the address accessed.
  \begin{enumerate}
    \item As a consequence, accessing an array leaks the index accessed.
  \end{enumerate}
  \item Any conditional statement leaks which branch was taken.
\end{enumerate}

Rule 1 is justified by theoretical concerns: a longer loop
uses more operations. In practice, it's difficult to observe
the iterations of each loop in a program from a global timing signal,
making this a pessimistic rule.

Rule 2 is justified by various cache based side-channels and attacks
\cite{
  bernstein_cache-timing_2005,
  yarom_cachebleed_2017,
  cabrera_aldaya_cache-timing_2019}.
Since caches only load information an entire line at a time,
this rule may seem too pessimistic. Perhaps only which cache line
was accessed should be kept secret \cite{brickell_technologies_2011}.
Unfortunately, it's possible to perform
attacks on a much finer level
\cite{
  bernstein_word_2013,
  osvik_cache_2006,
  yarom_cachebleed_2017}.
This is why we take a pessimistic position, and assume
that accesses leak their exact address.


Rule 3 is justified in two ways. First, if different branches of a conditional
statement execute a different number of operations,
we can inherently observe which branch was taken. Second, even if both
branches execute identitical operations, the CPU's branch predictor
can be exploited to leak information about which branch was taken
\cite{
  aciicmez_predicting_2006,
  aciicmez_power_2007,
  evtyushkin_jump_2016}.

In addition to these rules,
we need a basic set of trusted operations to build our programs.
We assume that addition, multiplication,
logical operations, and shifts, as implemented in hardware,
are constant-time in their inputs.
This is the case on most processors, one notable exception being
microprocessors
\cite{pornin_bearssl_nodate}. This assumption is reasonable
for the platforms targeted by Go and our library.

\subsection{Vulnerabilities in \texttt{big.Int}}

Go provides a general purpose type for Big Numbers: \texttt{big.Int}.
This type focuses on being optimized, and useful in various
situations. It does not focus on protecting against timing side-channels.
Unfortunately, out of convenience,
and for lack of better alternatives, it gets used in
cryptography, even inside of Go's standard library.

In this section, we look at some of the important implementation aspects
of \texttt{big.Int}, and how they might be vulnerable
according to our threat model.

\subsubsection{Padding}

The \texttt{big.Int} type normalizes numbers internally,
removing any leading zero limbs. Even if you initialize a number
using bytes zero-padded to a certain length, the resulting value
will immediately chop off these zeros.
By discarding them, operations on
this number will have fewer limbs to process, and will be faster.

Unfortunately, this means that \texttt{big.Int} pervasively leaks information
about the padding of numbers. Operations
take more time when a number has more limbs,
thus leaking the padding of numbers.
This has been exploited
in OpenSSL \cite{merget_raccoon_2019}, and might potentially
be a vulnerability in Go's cryptography library.

\subsubsection{Leaky Algorithms}

Because \texttt{big.Int} is not written with cryptography in mind,
its operations violate the rules in
\ref{threat_model}. Many methods take a different number of iterations
, branch conditionally, or
access memory differently
depending on their values. Because \texttt{big.Int}
is designed for general purpose use, this problem should only get worse
as the library is further developed and optimized.

Ultimately, the problem is not the existence of \texttt{big.Int},
but its use in Go's cryptography library, and in the broader ecosystem.

\subsubsection{Mitigations}

Although \texttt{big.Int} gets used in Go's cryptography library,
the authors are aware of its shortcomings, and have implemented
several mitigations to try and make its timing side-channels harder to
exploit.

One of the most important ones is a mitigation
for RSA: blinding \cite{kocher_timing_1996}.

To decrypt a ciphertext
$c = m^e \mod N$, we would normally calculate:

$$
c^d \mod N
$$

with $d$ our private key, and $(e, N)$ our public key.
When exponentiation is not implemented in a constant-time way, like
with \texttt{big.Int}, this process can leak information about $m$.
If an adversary can choose $c$, then this can leak information about
$d$ as well.

To mitigate this, instead of decrypting $c$ directly,
we first generate a random integer
$r \in (\mathbb{Z}/N\mathbb{Z})^*$. Then, we decrypt $r^e \cdot c$.
This gives us the value $r \cdot m$, and we can recover
$m$ by multiplying by $r^{-1}$.

While this effectively mitigates the simplest attacks
against exponentiation, a very leaky operation,
other methods are left unprotected, and may
have subtle exploits. For
example, we there are fundamental issues with
padding, which has lead to attacks
in OpenSSL \cite{merget_raccoon_2019}.

\section{Implementation}

We've implemented a library, called
\texttt{safenum} \cite{meier_cronokirbysafenum_2021}, intended to provide
a replacement for \texttt{big.Int}, suitable for cryptography.
To test its utility, we've replaced some
of \texttt{go/crypto}'s usage of \texttt{big.Int} with our own library,
in a separate repository
\cite{meier_cronokirbyctcrypto_2021}.

In this section, we go over the design and implementation of our
library.

\subsection{The \texttt{safenum} library}

Safenum defines a \texttt{Nat} type, intended to
replace \texttt{big.Int}. This type represents arbitrary
numbers in $\mathbb{N}$. Unlike \texttt{big.Int}, we do not handle
negative numbers. Handling a sign bit in constant-time is exceedingly
tricky. Thankfully, we haven't found this limitation to be restrictive
when replacing \texttt{big.Int} in Go's cryptography library.

We represent numbers in base $W := 2^{64}$. Concretely, we
store a number as a slice of type \texttt{[]uint}, in little
endian order. We call these the \enquote{limbs} of a number. For example,
the slice:
\begin{minted}{Go}
[]uint{13, 47, 52}
\end{minted}
represents the number:
$$
52 \cdot 2^{128} + 47 \cdot 2^{64} + 13
$$
These limbs might be padded, to conceal the true value of a number,
as we'll see later.

We provide operations for addition and multiplication of \texttt{Nats}.
We also provide numerous operations
for modular arithmetic, including modular addition, subtraction,
multiplication, exponentiation, inversion, reduction, and taking
square roots modulo prime numbers. We also provide operations
for serializing to and from bytes, as well as converting
to and from \texttt{big.Int}.

We try to structure the API in a similar way to \texttt{big.Int},
where an operation is performed on a separate
\texttt{Nat} receiving the result.
For example, this is the signature for
modular addition:

\begin{minted}{Go}
func (z *Nat) ModAdd(x *Nat, y *Nat, m *Modulus) *Nat
\end{minted}

This calculates $z \leftarrow x + y \mod m$, returning $z$. The advantage
of structuring the API this way, instead of simply returning a new
value, is that we can reuse the memory of $z$ for the result.

We go one step further, in fact, and use the memory of the receiving
\texttt{Nat} for all scratch space needed inside of an operation.
Structuring our operations this way limits memory waste.

\subsubsection{Handling Size}

A \texttt{big.Int} always stores a value using as few limbs
as possible. Its \textit{true size}, the number of limbs
necessary to store a value, always matches its \textit{announced size},
the number of limbs actually stored. When creating
a \texttt{big.Int}, any padding is immediately stripped away,
to satisfy this invariant.

For \texttt{Nat} however, we allow values to be padded.
This means that the announced size of a number
can be larger than its true size. Because the announced size
is what actually affects the runtime of our operations,
we can make sure that this size depends only on public information,
allowing us to keep the true size secret.

This means that every result we produce needs a clear
announced size. For modular
operations, we have an obvious choice: the size of the modulus.
When doing a modular operation, the result will always receive
the same announced size as the modulus.

For example, when doing modular addition:

\begin{minted}{Go}
func (z *Nat) ModAdd(x *Nat, y *Nat, m *Modulus) *Nat
\end{minted}

our result \texttt{z} will have the same announced size as \texttt{m}.
This implies that the true size of \texttt{z} is at most that
of \texttt{m}.
After modular addition, we know that $z \in [0, m - 1]$,
so this fact about the true size of \texttt{z} does
not reveal anything new.

When serializing a \texttt{Nat}, we respect its announced size,
and produce zeros for padding as necessary. This is done without
any special handling, since extra zeros are already stored
to match our announced size.

Similarly, when deserializing a \texttt{Nat} from bytes,
we respect any padding, unlike \texttt{big.Int}. For example,
if 32 big endian bytes are deserialized, we will end up
with a \texttt{Nat} with an announced size of 256 bits, regardless
of the value of those bytes.

This leaves us with non-modular addition and multiplication of numbers.
One approach is to use whatever size necessary to store
a maximal result.
For example, if we multiply
numbers $x_1$ and $x_2$, of announced sizes $b_1$ and $b_2$, then
our result will need a size of at most $b_1 + b_2$.

The disadvantage is that we use more and more space for each
operation, and we're unable to shrink the announced size,
even when we know that our result would fit.
Because of this, we opt towards letting users specify exactly how many
resulting bits they need in the output. For example,
multiplication has the following signature:

\begin{minted}{Go}
func (z *Nat) Mul(x *Nat, y *Nat, cap uint) *Nat
\end{minted}

Here \texttt{cap} is the number of bits that the result should have.
We use this to determine the result's announced length. Any output
beyond that capacity will simply be discarded.

In summary, the announced size of a \texttt{Nat} is always
clear based on how it's produced.
This size comes from deserializing
a value, from matching the size of a modulus, or from manually
deciding on an output size.

\subsubsection{Moduli}

In our library, we use a different type for
the moduli used in modular arithmetic: \texttt{Modulus}. There are several
reasons for doing this.

First, some modular operations use certain properties
of the modulus which can be pre-computed. For example, montgomery multiplication
uses $m^{-1} \mod W$, with $W$ our base.
Using a separate type for moduli lets us avoid
recomputing these properties, by storing them alongside
the value of our modulus.

Second, we allow the true size of a modulus to be leaked.
Moduli are stored \emph{without} padding.
This is desirable because modular reduction needs access to the most
significant bits of a modulus, and fetching this information without
leaking padding is exceedingly difficult. Furthermore,
by storing moduli without padding, the announced size of numbers produced
through modular operations is as tight as possible, making
every operation faster.

This assumption is safe in cryptography. Moduli are often public,
like with the public modulus $N$ in RSA. Since the exact value
is known, leaking the true size is fine.
There
are other cases where a private modulus is necessary.
Even then, the true size of that modulus remains known. For example,
when generating an RSA key, we know the factorization $N = pq$ of the modulus,
and calculate our private key modulo $\varphi(N) = (p - 1)(q - 1)$:

$$
d := e^{-1} \mod \varphi(N)
$$

Leaking the value of $\varphi(N)$ would be catastrophic. On the other hand,
it's clear that the true size of $\varphi(N)$ is approximately
that of $N$, which is known. In this case, leaking the true size of
$\varphi(N)$ is fine.

Finally, moduli are also allowed to leak whether or not
they are even. The motivation behind this is that certain modular
operations have faster variants for odd moduli, or require
different algorithms to support even moduli. Rather than
providing different methods for these different cases, we choose
to select the correct variant internally, allowing all modular
operations to work without exception. This requires
checking whether or not a modulus is odd or even, in order
to dispatch the right operation, leaking this bit of
information in the process.

Fortunately, we're not
aware of any realistic situation where the evennness of a
modulus needs to be kept secret. In fact, for the cryptographic
algorithms we know of, the evenness of a modulus 
is known statically, making this check fine.
In RSA, for example, the modulus $N$ is odd by construction, and
$\varphi(N)$ is even as a consequence.

Since moduli are stored without padding, we require a separate
type to not violate \texttt{Nat}'s contract around size.
\texttt{Nat} also has stronger constraints on information leakage,
so providing a separate type helps to avoid accidentally
violating them.

\subsection{Constant-Time Operations}

The rules we established in our threat model \ref{threat_model}
are quite stringent. For most operations, we need to have conditional
behavior depending on the values we process. Without access
to branching, this seems difficult.
Thankfully, we can emulate branching
without leaking information.
The idea is simple: to choose between two branches, we perform both,
and then combine
the results together, without revealing
which result is produced.

For example, a standard algorithm for modular subtraction would look
like this (in pseudo-Go):

\begin{minted}{Go}
func (z *Nat) ModSub(x *Nat, y *Nat, m *Modulus) *Nat {
  borrow := z.Sub(x, y)
  if borrow == 1 {
    z.Add(z, m)
  }
}
\end{minted}

The problem is that by conditionally adding in $m$, we reveal
whether or not $y > x$, and a borrow occurred. Our solution
uses a new primitive:

\begin{minted}{Go}
func (z *Nat) ctCondCopy(v choice, y *Nat) *Nat
\end{minted}

This function assigns \texttt{y} to \texttt{z} if
\texttt{v == 1}, and does nothing otherwise.
Furthermore, this primitive should leak no information about
the condition.

With this in place, we can implement modular subtraction without leakage:

\begin{minted}{Go}
func (z *Nat) ModSub(x *Nat, y *Nat, m *Modulus) *Nat {
  borrow := z.Sub(x, y)
  scratch := new(Nat).Add(z, m)
  z.ctCondCopy(choice(borrow), scratch)
}
\end{minted}

We always perform the addition, and copy over the result if necessary,
without leaking the value of \texttt{borrow}.

This pattern is what allows us to replace
branching with constant-time operations throughout
our algorithms.

\subsubsection{Building Primitives}

How do you make primitives like
\texttt{ctCondCopy}, which let you choose results without
leaking which choice was made?

The methods for constant-time choice are analogous to
those used for variable-time choice. Instead of using
\texttt{bool} to represent the result of a condition, we use

\begin{minted}{Go}
type choice Word
\end{minted}

The value of a choice is either \texttt{1} or \texttt{0},
but we use the same
type as full limbs, to avoid having the compiler
re-insert branches into our code.

From this choice value, we can build a primitive that selects
between two limbs, without leaking which choice was selected:

\begin{minted}{Go}
func ctIfElse(v choice, x, y Word) Word {
  mask := -Word(v)
  return y ^ (mask & (y ^ x))
} 
\end{minted}

This routine returns \texttt{x} if
\texttt{v == 1}, and \texttt{y} otherwise.
The bitwise operations don't leak information about
our choice, unlike a conditional statement.

If \texttt{v == 0}, then \texttt{mask} contains only zeros, and we're left
with \texttt{y}. When \texttt{v == 1},
then \texttt{mask} only contains ones. The \texttt{y}s cancel each
other out,
leaving us with \texttt{x}.

We can use this primitive to build up a larger selection primitive,
allowing us to conditionally
assign an entire slice of limbs to another:

\begin{minted}{Go}
func ctCondCopy(v choice, x, y []Word) {
  for i := 0; i < len(x); i++ {
    x[i] = ctIfElse(v, y[i], x[i])
  }
}
\end{minted}

These primitives allow us to use \texttt{choice} to introduce conditional
behavior without leaking our choices, but we also need
a way to
create \texttt{choice} values. These are built up
in a similar way, from small functions to larger ones.

We can decide whether or not two limbs are equal using some bitwise
trickery:

\begin{minted}{Go}
func ctEq(x, y Word) choice {
  q := uint64(x ^ y)
  return 1 ^ choice((q|-q)>>63)
}
\end{minted}

To understand why this trick works, first realize that in two's complement,
if you take any number $x \neq 0$,
then the most significant bit of either $x$ or $-x$ is set.
For $0$, neither are set.
We can check that \texttt{q} is non-zero by seeing
if either bit is set:
\begin{minted}{Go}
choice((q|-q)>>63)
\end{minted}
Since \texttt{q} is zero precisely when \texttt{x} and \texttt{y}
are equal, we negate this non-zero check.

We can use this primitive to compare entire slices of limbs
in constant time:

\begin{minted}[samepage]{Go}
func cmpEq(x []Word, y []Word) choice {
  res := choice(1)
  for i := 0; i < len(x) && i < len(y); i++ {
    res &= ctEq(x[i], y[i])
  }
  return res
}
\end{minted}

Two slices are equal when each of their limbs match.
Since we can't exit early without leaking information
about \texttt{x} and \texttt{y}, we implement this formula
using bitwise operators, without any short-circuiting.

\subsection{Algorithm Choices}

While going over how each operation works in detail is outside
the scope of this report, describing some the high-level
techniques used for these operations is still valuable.

Many of these operations were inspired by the excellent
work of Thomas Pornin in BearSSL
\cite{pornin_bearssl_2020-1}.

\subsubsection{Modular Reduction}

To reduce a number modulo $m$, we first implement an operation
that allows us to shift in a single limb. This
lets us reduce a number
of the form:
$$
z := a \cdot W + b
$$
with $a \in [0, m - 1]$ and $b \in [0, W - 1]$. We can use
this operation to reduce an arbitrary $x$. First, we can
write out
$x$ in base $W$:
$$
x = x_n W^n + x_{n -1}W^{n - 1} + \cdots + x_0
$$
Then, we can rewrite this expression to see how
shifting plays a role:
$$
x = ((x_n W + x_{n - 1})W + \cdots)W + x_0
$$
Having written $x$ in this form, it becomes clear that to reduce
$x$ modulo $m$, we can use our shifting primitive, folding
in the limbs of $x$ from most significant to least significant.

Implementing the shifting operation needs
the quotient $q := \lfloor z / m \rfloor$,
in order to calculate the remainder $z - q m$.
Instead of calculating $q$ directly, we instead
produce an estimate $\hat{q}$, by using the most significant
$64$ bits of $m$ to divide $128$ corresponding bits of $z$.
This estimate satisfies $\hat{q} = q \pm 1$, so we then
conditionally add and subtract $m$ to correct
for this error.
This technique is further described in \cite{pornin_bearssl_2020-1}.

\subsubsection{Inversion}

For modular inversion, which method we use depends on whether
our modulus is odd or even. As mentioned previously,
we're allowed to choose the right method dynamically.

For odd moduli, we use
the binary GCD algorithm, as described in \cite{pornin_optimized_2020}.
We have yet to implement the most optimized version, which accumulates
intermediate results into single limb registers. Instead, we
perform full width operations at each iteration.

For an even modulus $m$,
there's a standard trick to calculate $x^{-1} \mod m$.
First, we calculate
$u := m^{-1} \mod x$,
using our method for odd moduli,
and then we calculate our desired inverse as:

$$
\frac{um - 1}{x}
$$

This requires implementing a division operation that works
with padded numbers. We can't reuse the
limb-by-limb modular reduction
routine we normally use, and instead adapt
a bit-by-bit reduction routine, as described in
\cite{pornin_bearssl_2020-1}.

\subsubsection{Exponentiation}

For exponentiation, we use left-to-right exponentiation,
with a window size of 4 bits. To perform window lookups,
we use a constant-time conditional assignment
over each of the possible window values. Even though this requires traversing
the entire window table at every iteration, we've found that this method
is still faster than using a smaller window size of 2 bits.

\subsubsection{Multiplication}

For modular multiplication, we calculate the full product
$ab$, and then reduce this result modulo $m$. This works
for both odd and even moduli.

In the case of odd moduli, using Montgomery multiplication
\cite{kaya_koc_analyzing_1996-1, pornin_bearssl_2020-1} is
faster, but requires converting inputs into their Montgomery
representation, producing outputs in this representation.
For exponentiation,
the cost of converting to and from this representation is amortized,
since we do many multiplications between conversions, making
it considerably faster to use this technique. Unfortunately,
Montgomery multiplication doesn't work for even moduli,
so exponentiation is slower in that case.

\subsubsection{Modular Square Roots}

To calculate $\sqrt{z} \mod p$, we assume that $p$ is prime.
Calculating square roots with a composite modulus is,
in general, as hard
as factoring it \cite{buhler_basic_2008}, which is why
we only support prime moduli.
Which algorithm we use depends on whether $p = 3 \mod 4$,
or $p = 1 \mod 4$.

In the first case, we can calculate a root as:
$$
\sqrt{z} = z^{(p + 1) / 4} \mod p
$$

In the second case, we use a constant-time variant
of the Tonnelli-Shanks algorithm, as described in 
\cite{wahby_hashing_nodate}.

\subsection{Implementation Choices}

In this section, we describe a few more details
about how numbers are stored, and the motivation behind
these choices.

\subsubsection{Saturated or Unsatured Limbs}

We store numbers in base
$W := 2^{64}$. This means that we use the full width of a register
to store each limb. Because of this, we say that are limbs
are \textit{saturated}. It's also possible to store
limbs \textit{unsaturated}, by using fewer than $64$ bits.
BearSSL 
\cite{pornin_bearssl_2020-1}
takes this approach, using only $31$ bits of the available $32$ bits
in the integers it uses. For $64$ bit registers, using $63$ bit unsaturated
limbs would be the analogous choice.

There are two compelling reasons for using unsaturated limbs.

First, this leaves an extra bit of space to hold a carry
or borrow after an addition or subtraction. This allows
us to chain together carries to implement operations over multiple
limbs, without having to use assembly instructions. In Go,
this isn't a concern, since \texttt{bits.Add} and
\texttt{bits.Sub} can be used to implement these intrinsics
in a cross-platform way.

Second, if we use $w$ bits for each limb, then montgomery
multiplication needs to work with a value of size $2w + 1$ bits. With a fully
saturated limb of $64$ bits, we need $129$ bits. This uses
an extra register compared to unsaturated limbs of $63$ bits. Because
montgomery multiplication is called very often during exponentiation,
this can yield considerable savings.

One disadvantage of using unsaturated limbs comes when converting
numbers to and from bytes. With fully saturated limbs, our
$64$ bit limbs are composed of exactly $8$ bytes. With $63$ bit limbs,
this isn't the case, making conversion more
complicated and expensive.

Using unsaturated limbs would
also require storing additional information about the exact
announced size of a number, instead of being able to use
the number of limbs directly, as we do now.

Finally, by using saturated limbs, we can use the assembly routines
already implemented for low level operations
in \texttt{big.Int}, which uses saturated limbs. These low-level
routines are constant-time, and speed up basic operations.

Interoperability with these routines
is ultimately the main reason why we opted
for using saturated limbs.

\subsubsection{Redundant Reductions}

Our library tries to prevent misuse. Because of this,
modular operations work even if their inputs are not already
reduced. For example, addition modulo $m$ should return the
right result, even if the inputs are greater than $m$.
Unfortunately, the cost of reducing inputs modulo $m$ when
they are already in range is not desirable, since this operation
is relatively expensive. Ideally, we'd like to avoid reducing
inputs when we know this reduction
is redundant.

To implement this, each \texttt{Nat} stores a pointer to a modulus,
indicating that it has been reduced by this modulus.
When we reduce a number modulo $m$, we check this pointer,
and skip the reduction if it matches $m$. If we modify
the value of a number, we update the modulus it points to
accordingly.
For example:
\begin{minted}{Go}
z.ModAdd(x, y, m)
\end{minted}
will set \texttt{z}'s modulus to \texttt{m}, and calling:
\begin{minted}{Go}
z.SetBytes(data)
\end{minted}
will clear \texttt{z}'s modulus, making it \texttt{nil}.

We modify this pointer based strictly on what methods are called,
never on the actual value of a result. Thus, the dynamic
checks of this pointer
only depend on the call-graph of our program.
Since this graph is statically determined, these redundant reduction
checks don't impact the constant-time properties of our library.

\section{Results}

We've compared the performance of our library
with \texttt{big.Int}, operation by operation, as well as in the
context of the \texttt{go/crypto} package.
Overall, our library is about 2.6x slower than using
\texttt{big.Int} for most operations,
but only 2x slower in realistic situations.
In this section, we
present these results in detail.

\subsection{Comparison with \texttt{big.Int}}

We've set up a series of benchmarks to compare the performance
of \texttt{Nat} compared to \texttt{big.Int} on various operations.

The following operations are all implemented on values, exponents,
and (odd) moduli of 2048 bits. For raw addition and multiplication,
we use the full size necessary to represent the result in our benchmarks.

\begin{center}
 \begin{tabular}{|c | c | c | c|} 
 \hline
 Operation & op / s (\texttt{big.Int}) & op / s (\texttt{Nat}) & ratio \\ [0.5ex] 
 \hline\hline
 Addition & 10,980,842 & 12,164,599 & 0.90 \\
 \hline
 Modular Addition & 6,986,739 & 3,075,188 & 2.27 \\
 \hline
 Multiplication & 1,316,322 & 542,385 & 2.43 \\
 \hline
 Modular Reduction & 454,917 & 63,253 & 7.19 \\
 \hline
 Modular Multiplication & 1,000,000 & 44,596 & 22.42 \\
 \hline
 Modular Inversion & 1,000,000 & 621 & 1610 \\
 \hline
 Modular Exponentiation & 223 & 86 & 2.59 \\
 \hline
\end{tabular}
\end{center}

The most expensive operation, by far, is exponentiation. Because
of this, it's fair to compare the performance on these two types
mainly on this operation. We can see that \texttt{Nat} is 2.6x slower
compared to \texttt{big.Int} for exponentiation, although
some operations are much slower.

For comparing modular square roots, we used the primes
$p_3 = 2^{244} + 79$, which is $3 \mod 4$,
and $p_1 = 2^{244} + 153$ which is $1 \mod 4$. We use different primes
to test the various codepaths for modular square roots:

\begin{center}
 \begin{tabular}{|c | c | c | c|} 
 \hline
 Operation & op / s (\texttt{big.Int}) & op / s (\texttt{Nat}) & ratio \\ [0.5ex] 
 \hline\hline
 $\sqrt{z} \mod p_3$ & 40,464 & 26,886 & 1.50 \\
 \hline
 $\sqrt{z} \mod p_1$ & - & 7,867 & - \\
 \hline
\end{tabular}
\end{center}

We couldn't find a large value where Go's Tonneli Shanks routine
managed to find a square root without hanging, although we
expect the ratio to be similar to the other case.

\subsection{Comparison with \texttt{go/crypto}}

We've created a forked package
\cite{meier_cronokirbyctcrypto_2021}
of \texttt{go/crypto}, where we've replaced \texttt{big.Int}
with our own \texttt{Nat} type for both RSA and DSA. All of the code
using \texttt{big.Int} has been replaced, with the exception of
primality checking. This demonstrates the utility of
our package for writing cryptographic code.

We've also run benchmarks to assess the performance impact, as we
show in the following table:

\begin{center}
 \begin{tabular}{|c | c | c | c|} 
 \hline
 Operation & op / s (\texttt{big.Int}) & op / s (\texttt{Nat}) & ratio \\ [0.5ex] 
 \hline\hline
 RSA Decrypt & 670 & 312 & 2.15 \\
 \hline
 RSA Sign & 675 & 372 & 1.81 \\
 \hline
 RSA Decrypt (3 Prime) & 1173 & 596 & 1.97 \\
 \hline
 DSA Sign & 6202 & 2625 & 2.36 \\
 \hline
 DSA Parameters & 0.89 & 1.64 & 0.54 \\
 \hline
\end{tabular}
\end{center}

We use a 2048 bit modulus for both RSA and DSA. For RSA, we use
the CRT optimization, instead of using exponentiation directly.
Our benchmarks for \texttt{big.Int} use blinding, but our benchmarks
for \texttt{Nat} do not. Because \texttt{Nat} is constant-time,
we don't need to use blinding to mitigate timing attacks.
We don't include DSA verification, since this can be safely done
with \texttt{big.Int}.

Overall, we can see that in a real world scenario, the use
of \texttt{Nat} is only 2x slower. This can surely be improved,
but is already an encouraging result.

\section{Further Work}

While we're happy with the utility of our library,
and the performance results we've managed to achieve, it's
of course still possible to improve on this front.

\subsection{Verifying Constant-Time Properties}

Ultimately, we would like to have more assurance about the
constant-time properties of our library. Our code hasn't
undergone an audit, nor have we verified the assembly output
produced by the Go compiler to ensure that it meets our demands.

Ideally, it would be nice to incorporate some kind of automated
analysis of our code to detect timing side-channels. An approach
similar to dudect
\cite{reparaz_dude_2017} 
might be an interesting way to provide a form of fuzz testing
to detect unwanted time-variation.

\subsection{Optimizing Assembly Routines}

Currently, we rely on some assembly routines pulled from
\texttt{big.Int}, slightly modified to avoid jumping to variable-time
routines. Unfortunately, not all of the primitive operations we would
like to have are present. Furthermore, we could reduce memory usage
in some places, by having these operations present a "conditional"
variant. For example, we could have an add operation taking a
\texttt{choice} flag, allowing us to choose whether or not to perform
an addition, without leaking information. This would avoid having
to use a scratch buffer and a conditional copy.

To gain similar speed to the other primitives, these new primitives
would also need to be implemented in assembly. This would be time-consuming,
but likely worth the effort. There are also new solutions
to help with writing assembly routines in Go, such as the Avo library
\cite{mcloughlin_mmcloughlinavo_2021}.

\subsection{Upstreaming to \texttt{go/crypto}}

While we hope our library is immediately useful for the broader
ecosystem, it's not realistically going to be replacing
\texttt{big.Int} in Go's cryptography library any time soon.

The most likely path towards removing \texttt{big.Int} from
\texttt{go/crypto} is to move towards specialized arithmetic
implementations for each prime field involved in ECC. DSA is a legacy
algorithm, where the security flaws introduced
by \texttt{big.Int} are not of major concern.

This leaves RSA. Unfortunately, because RSA requires dynamic
moduli, we need a Big Number library of some kind.
Ideally, this library would be internal to RSA, allowing constant-time
operation, and severing the bridge between Go's cryptography
package, and \texttt{big.Int}.

We've
submitted a patch \footnote{\url{https://go-review.googlesource.com/c/go/+/326012}}
for Go's implementation of RSA,
replacing \texttt{big.Int} with an internal number
type,
using the minimal amount of code
necessary to implement encryption and decryption.
The public API, as well as key generation,
still use \texttt{big.Int}.

Using unsaturated limbs, we've found that our version of RSA
suffers only a 1.7x slowdown, while implementing encryption and
decryption in constant-time.

\section{Conclusion}

In summary, we have shown
why Go's general purpose big number type, \texttt{big.Int},
is not suitable for Cryptography.
Unfortunately, this type gets used out of convenience,
and for lack of better alternatives,
even in Go's own cryptography library.

To address this, we've created a replacement library for \texttt{big.Int},
achieving a slowdown of only 2.6x for most operations, while attempting
to provide constant-time operation.

To test the utility of this library, we've replaced the usage
of \texttt{big.Int} in Go's implementation of RSA, and DSA, and found
only a slowdown of 2x.

\section*{Acknowledgements}

Firstly, I'd like to thank Professor Bryan Ford for supervising this work.
I'd like to thank Pierluca Borsò as well, for letting me 
work on this project.
Finally, I'd also like to thank Daniel Huigens, and Marin Thiercelin,
from ProtonMail, for their advice and industry perspective on this work.

\addcontentsline{toc}{section}{Acknowledgements}

\bibliographystyle{plainurl}
\bibliography{references}
\end{document}
